\documentclass[a4paper, notitlepage, twoside]{article}

\usepackage[utf8]{inputenc}
%\usepackage[T1]{fontenc} %génére une police horrible sur linux
\usepackage[french]{babel}

\pagestyle{plain}
\begin{document}

Projet : Application Android Anti-Masque

\section{Objectifs :}
\begin{itemize}
	\item produire une interface graphique sous Android
	\item utilisation du DragNDrop au sein de cette interface
	\item envoyer les coordonnées en temps réel au serveur en utilisant le protocole OSC
\end{itemize}

\section{Développement :}
Le développement de l'application se divise en 3 partie distinctes :
\\ \\
\noindent Interface :
\begin{itemize}
	\item afficher une IHM (= Interface Homme-Machine) simple permettant de manipuler la position des points A, B, C et D
\end{itemize}
Manipulation des points :
\begin{itemize}
	\item lors de la modification de la position d'un des points, obtenir en pourcentage la position du point
	\item factoriser la position du point
	\item envoyer la nouvelle position du point au serveur en utilisant le protocole OSC
\end{itemize}
	Amélioration de l'interface :
\begin{itemize}
	\item améliorer l'interface en un polygon à 4 sommet avec un cercle sur chaque sommet, chaque sommet équivaut à un point
	\item déplacer au "toucher", un des sommets indépendament des autres
\end{itemize}

\section{Technique :}
\begin{itemize}
	\item protocole de communication client-serveur : OSC
	\item adresse de communication : 10.0.1.100/24
	\item port de l'application : 9876 en UDP
\end{itemize}


\section{Exemples :}
\begin{itemize}
	\item /instruction = P(x, y)
	\item /mallarme/masque/A 0.5 0.5 = A(0.5, 0.5)
	\item /mallarme/masque/C 0.1 0.9 = C(0.1, 0.9)
\end{itemize}

\end{document}
